\section{Introduction}
% Computer Society journal papers do something a tad strange with the very
% first section heading (almost always called "Introduction"). They place it
% ABOVE the main text! IEEEtran.cls currently does not do this for you.
% However, You can achieve this effect by making LaTeX jump through some hoops
% via something like:
%
%\ifCLASSOPTIONcompsoc \noindent\raisebox{2\baselineskip}[0pt][0pt]%
%{\parbox{\columnwidth}{\section{Introduction}\label{sec:introduction}%
%\global\everypar=\everypar}}% \vspace{-1\baselineskip}\vspace{-\parskip}\par
%\else \section{Introduction}\label{sec:introduction}\par \fi
%
% Admittedly, this is a hack and may well be fragile, but seems to do the trick
% for me. Note the need to keep any \label that may be used right after
% \section in the above as the hack puts \section within a raised box.



% The very first letter is a 2 line initial drop letter followed by the rest of
% the first word in caps (small caps for compsoc).
% 
% form to use if the first word consists of a single letter:
% \IEEEPARstart{A}{demo} file is ....
% 
% form to use if you need the single drop letter followed by normal text
% (unknown if ever used by IEEE): \IEEEPARstart{A}{}demo file is ....
% 
% Some journals put the first two words in caps: \IEEEPARstart{T}{his demo}
% file is ....
% 
% Here we have the typical use of a "T" for an initial drop letter and "HIS" in
% caps to complete the first word.
%\IEEEPARstart{T}{his}

At 6:15 pm on the 3rd of January, 2009 UTC Satoshi Nakamoto (a pseudonym)
created and announced block of data known as the Genesis Block of the Bitcoin
block chain\cite{satoshi}. With this block, the first 50 units of the decentralised peer to
peer currency Bitcoin were created.  This currency was described by Satoshi in
his 2008 paper

Now over ten million bitcoins have been generated, and are currently trading at
\input{price}.

The key difference between bitcoin and other digital currencies such as e-gold
or PayPal is that it does not have a central authority, transactions are
irreversible, and transactions can be made pseudo anonymously.  The bitcoin
protocol allows multiple different clients to connect, verify and disseminate
transactions.  The reference bitcoin client, has received multiple releases and
is now known as Bitcoin-Qt.

\begin{itemize} \item The need for bitcoin \item What this paper is about - The
research on bitcoin, which has focussed on Anonymity, Economics and proposals
for improvements \end{itemize}

\section{Background} The core of any currency is to allow users to participate
in transactions that result in the transfer of value between the participants.
If Alice wishes to send some value to Bob, Bob will need to know that Alice has
the funds to send that transaction and has not sent that money to anyone else.
This problem is easily solvable in the physical world where the expense to
duplicate or create more of a resource can be ensured to be more than the value
of the item: for example creating counterfeit paper currency or creating a
commodity like gold.  This property of physical goods brings with it another
problem: moving the value over a distance, this can be very expensive
especially with highly massive, solid commodities like gold.  A digital
currency aims to recreate the difficulty in duplication, while taking advantage
of the ease of transfer that digital information brings.

Some solutions to these problems have already been discussed in the scientific
literature: all systems assume a trusted central authority as part of the
protocol. Usually to maintain a record of transactions performed and to mint
currency.

\subsection{Central Banking} TODO: Systems such as PayPal or Facebook Credits
in which only the central authority of the currency can create transactions.

\subsection{Unique Token} Cite: Medvinsky and Neuman discuss one solution,
NetCash with a similar protocol implemented under the name opencoin, uses a
unique token held by both the central authority and the owner of the coin. To
create currency the user and central authority agree on a unique token in
exchange for some currency or resource. When that user wishes to transact value
to another, they simply send the token to the other user.  To verify that the
transaction is legitimate, other receiving users send the token to the
exchange, to be swapped for a new spendable token meaning the original token
can no longer be spent. One issue with this protocol is the central authority
cannot be independently verified, as such the authority could choose to deny
knowledge of a token, or allow multiple users to claim the same token.

\subsection{Digital signature chain} An alternative solution proposed operates
``using a chain of digital signatures'' where transaction messages are created
using public key cryptography and the chain of digital signatures can be traced
back to the original creation of the unit of value. For example Alice
cryptographically signs a message ``I pay 5 coins to Bob''.  This system presents
the same problems as before regarding value distribution and potential double
spending i.e. how does Bob know that Alice has the value to start with and how
does Bob know that this value hasn't already been spent. In [Cite] proposal the
solution is to involve a central authority that hosts a public copy of every
transaction sent and is trusted to create and allocate value fairly such as on
receipt of a payment in fiat currency, or some resource.

Bob can verify that Alice was given some value from a chain of transactions
leading back to the central authority where the value is assumed to be created
and can subtract the total value of Alice’s spent transactions to determine if
the transaction sent to Bob is valid.  While this system prevents the central
authority from forcing a user to spend money, unlike token based digital
currencies or central banking, the central authority still holds a significant
position of power and thus demands a great deal of trust from the user that the
authority is both benign and secure. For example the central authority could
cooperate with, or more simply act as, a user attempting to achieve a double
payment attack by showing different views to different users. Because the
authority is the sole party involved in the minting process that authority can
create more units of currency.

[diagram of a chain of value movement]

[diagram of double spend]

In both systems where a central authority exists, while it may be reasonable to
trust some agent for the purpose of authenticating the ownership of value that
authority may act in a way outside of the agents’ control.  The server hosting
the system could be compromised either through technical means or through the
use of force, as in the case of E-Gold.

\begin{itemize} \item Show different views to different users, resulting in a
double spend \item Allocate money unfairly or unexpectedly, with excessive
minting or ``easing'' \item Be shut down by a government \item Cryptographic
signing \item Transactions as cryptographic signatures \item The double
spending problem \item HashCash \end{itemize}

\subsection{The Bitcoin Protocol} Bitcoin extends the digital signature chain
[hyperref] based transaction system and replaces the central authority with a
distributed, decentralized timestamp consensus that is used to agree on the
order that transactions occurred in.  Any user can, on receipt of a
transaction, query the time stamp server and determine if that transaction has
been invalidated by a prior transaction.  Distributed Transaction Validation

The time stamp server creates a chain of blocks of transactions, each referring
to the previous by hash value.  The transactions to be included in the chain
are discovered when peers broadcast them into a peer discovery network that
operates in a similar manner to a BitTorrent swarm.  Any user can contribute to
this chain of blocks, but before another peer will relay it to others, it must
meet a ``proof-of-work'' requirement: the SHA256 hash value of each block must
not exceed a value determined by the current ‘difficulty’.  Because the SHA2
hashing function has been designed such that the value of an amount of data is
unpredictable and evenly distributed in the range of the function, the only way
for a participant to create data with a hash value below the difficulty limit
is to repeatedly try different input data values. Transaction block data can be
altered without changing the transaction semantics because the transaction
block format includes a ‘nonce’ field for this purpose.

Because each block refers to the previous block, ``To modify a past block, an
attacker would have to redo the proof-of-work of the block and all blocks after
it''.  As more blocks are created the more expensive it is for any group to
alter the consensus reached by the peer network.  Because the proof of work
system is computationally expensive an incentive, other than the increased
certainty of any transaction a user has previously received will remain valid,
is included.  While all the transactions included in a block must have at least
the input value as output value, one transaction can be included with no input
and an output dependant on the total number of blocks due to halve every 4
years. Currently this reward is 25 bitcoins. This reward is paid for by the
consensual deflation of every participants held bitcoin. This process also
serves to solve the distribution problem of digital currencies.

\begin{itemize} \item Public keys and addresses.  \item How bitcoin aims to
solve the double spending problem without a central authority \item Bitcoin
clients present public keys to the user as ``addresses'', these are the result of
the hash of a public key and a checksum. Similar in purpose as the Luhn check
on credit cards, this process is used to reduce the likelihood of losing
bitcoin to a public key with no paired private key by preventing mistakes when
entered into clients by hand.  \item Transaction types \item Change address.
The change address is always a new address \item TestNet bitcoin \end{itemize}

\section{Anonymity} As part of the Bitcoin protocol, all transactions must be
broadcast publicly however, Satoshi maintains that ``privacy can still be
maintained by breaking the flow of information [...] by keeping public keys
anonymous''.  Each Bitcoin address can be considered one pseudonym of many of
the owner of the associated private key, because each person can own multiple
Bitcoin addresses in a wallet. Satoshi asserts that because of this disconnect
between user and address it is difficult to associate a particular transaction
with a particular user.

\subsection{Taint Analysis} Harrigan and Reid, considers the feasibility of
maintaining this pseudonymity in practise, by investigating the multiple input
semantics of the Bitcoin transaction format\cite{reid-anon}. 

Harrigan and Reid argue that any key taking part in this process was in a
single wallet, and thus owned by a single user. Therefore by combining the
knowledge of how keys are grouped into wallets with knowledge from outside the
protocol other inferences can be made, with the aim to determine the identity
of the user behind the wallet, and thus any transactions made.

Harrigan and Reid aim to de-anonymize a user using two phases.  First to
contract the publicly available, as published in the blockchain, ``transaction
network'' to a ``user network” by determining groups of addresses that appear to
behave as part of a single wallet.  The second phase is to  link that group
using publicly available ``Off-Network Information'' linking addresses to keys,
eg by crawling the web for forum posts with including both user-names and
addresses.

Multi-input transactions are used to distinguish the behavior of a group of
keys from any other in the transaction network.

In the Bitcoin transaction format, if a user cannot fulfil the value of a
transaction by spending a single previous output, the client will combine
multiple previous outputs where the user has the ability to fulfill their spend
conditions.  When combined the Bitcoin client must fulfill the spend conditions
of all of the outputs, usually the conditions result in the requirement to sign
the transaction with all of the participating private keys.

[diagram of multi input transactions grouping wallets]

The Bitcoin community refers to this process as ``Taint Analysis'' [cite].

\begin{itemize} \item Briefly mentions mixers and ``repeatedly sending fractions
of their Bitcoins to themselves using multiple (newly generated) public-keys''
Although that does not defeat taint analysis as these can be assumed to be
owned by the originating wallet until otherwise linked.  \item Change
addresses, bug that lead to them being in the same position each time.
Androulaki et al, note that change addresses are clearly distinguishable by
being the only newly generated address in the output of the transaction.
However it is recommended to newly generate a Bitcoin address for each
receiving transaction.  \end{itemize}

\subsection{Taint Analysis Disruption} \subsubsection{ZeroCoin} Miers et al.
propose a solution to the de-anonymization proposed in taint analysis, by
defining new transaction types that take advantage of zero knowledge
proofs\cite{zerocoin}.
\subsubsection{Multi-Input Transaction}

In his forum post, ``I taint rich!'', Bitcoin developer Gregory (gmaxwell)
Maxwell describes a process to disrupt ``brain-dead automated (taint) analysis''.
Because the conditions of a set of previous outputs can be fulfilled partially,
multiple users can cooperate to create a single transaction that causes their
public keys to appear linked in a single wallet to taint analysis, without
revealing their private keys to the group.  As such Satoshi’s claim, and key
assumption in Harrigan and Reid’s paper, that ``Some linking is still
unavoidable with multi-input transactions, which necessarily reveal that their
inputs were owned by the same user'' is wrong\cite{satoshi}.

While in the example described it is clear how the addresses are really linked,
due to the cooperation being performed on a public forum, the same address used
on the input and output and the fact that that address is the vanity address,
``1gmaxw''. The complexity of this process means it is not well used.

For a serious attempt to disrupt taint analysis the taint disrupting
transaction must have the following properties:  Each output must be to an
address that has not been used previously. There must be an even number of
bitcoin sent to each output. The transaction must not be traceable to a network
address.  The communication during cooperation must not be intercepted. The
communication must not be shown to have taken place.  At this point, from a
view of the blockchain, the two wallets appear linked, under the
multi-transaction heuristic and change address heuristic. From the view of the
cooperating peers, however, it is clear the true owner of the output keys as
such this process must take place repeatedly between many different peers.

\subsubsection{A design for an automatic taint analysis disruption}

\begin{itemize} \item Using the Tor hidden service \item Some sort of central
or distributed registration system for peers \item A peer may connect to any
other peer, each peer creates a new address, they both determine an equal
amount of bitcoin to send. And create a multi-input transaction from both
users, split equally into the new addresses.  \item Assuming that the ownership
of originating addresses was known publicly the ownership of the two new
addresses is now 50/50.  \item Diagram \item Repeat the process \end{itemize}

\section{Improvements To the Bitcoin system}

\begin{itemize} \item What is a forking change?  \item What improvements are
possible \item Not strictly fully decentralized, concept of consensus,
checkpointing.  \item The hash power voting system \end{itemize}


% An example of a floating figure using the graphicx package.  Note that \label
% must occur AFTER (or within) \caption.  For figures, \caption should occur
% after the \includegraphics.  Note that IEEEtran v1.7 and later has special
% internal code that is designed to preserve the operation of \label within
% \caption even when the captionsoff option is in effect. However, because of
% issues like this, it may be the safest practice to put all your \label just
% after \caption rather than within \caption{}.
%
% Reminder: the "draftcls" or "draftclsnofoot", not "draft", class option
% should be used if it is desired that the figures are to be displayed while in
% draft mode.
%
%\begin{figure}[!t] \centering \includegraphics[width=2.5in]{myfigure} where an
%.eps filename suffix will be assumed under latex, and a .pdf suffix will be
%assumed for pdflatex; or what has been declared via
%\DeclareGraphicsExtensions.  \caption{Simulation Results.} \label{fig_sim}
%\end{figure}

% Note that IEEE typically puts floats only at the top, even when this results
% in a large percentage of a column being occupied by floats.  However, the
% Computer Society has been known to put floats at the bottom.


% An example of a double column floating figure using two subfigures.  (The
% subfig.sty package must be loaded for this to work.) The subfigure \label
% commands are set within each subfloat command, and the \label for the overall
% figure must come after \caption.  \hfil is used as a separator to get equal
% spacing.  Watch out that the combined width of all the subfigures on a line
% do not exceed the text width or a line break will occur.
%
%\begin{figure*}[!t] \centering \subfloat[Case
%I]{\includegraphics[width=2.5in]{box}% \label{fig_first_case}} \hfil
%\subfloat[Case II]{\includegraphics[width=2.5in]{box}%
%\label{fig_second_case}} \caption{Simulation results.} \label{fig_sim}
%\end{figure*}
%
% Note that often IEEE papers with subfigures do not employ subfigure captions
% (using the optional argument to \subfloat[]), but instead will
% reference/describe all of them (a), (b), etc., within the main caption.


% An example of a floating table. Note that, for IEEE style tables, the
% \caption command should come BEFORE the table. Table text will default to
% \footnotesize as IEEE normally uses this smaller font for tables.  The \label
% must come after \caption as always.
%
%\begin{table}[!t] % increase table row spacing, adjust to taste
%\renewcommand{\arraystretch}{1.3} if using array.sty, it might be a good idea
%to tweak the value of \extrarowheight as needed to properly center the text
%within the cells \caption{An Example of a Table} \label{table_example}
%\centering % Some packages, such as MDW tools, offer better commands for
%making tables % than the plain LaTeX2e tabular which is used here.
%\begin{tabular}{|c||c|} \hline One & Two\\ \hline Three & Four\\ \hline
%\end{tabular} \end{table}


% Note that IEEE does not put floats in the very first column - or typically
% anywhere on the first page for that matter. Also, in-text middle ("here")
% positioning is not used. Most IEEE journals use top floats exclusively.
% However, Computer Society journals sometimes do use bottom floats - bear this
% in mind when choosing appropriate optional arguments for the figure/table
% environments.  Note that, LaTeX2e, unlike IEEE journals, places footnotes
% above bottom floats. This can be corrected via the \fnbelowfloat command of
% the stfloats package.



\section{Conclusion}
